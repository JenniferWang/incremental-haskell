\documentclass[a4paper, twocolumn]{article}
\usepackage{geometry}
\geometry{left=2.5cm,right=2.5cm,top=2.5cm,bottom=2.5cm}

\usepackage{graphicx}
\usepackage{natbib} 
\usepackage{amsmath} 
\usepackage{listings}
\usepackage{authblk}
\usepackage{color}

\definecolor{dkgreen}{rgb}{0,0.6,0}
\definecolor{gray}{rgb}{0.5,0.5,0.5}
\definecolor{mauve}{rgb}{0.58,0,0.82}
\lstset{frame=tb,
  language=Haskell,
  aboveskip=3mm,
  belowskip=3mm,
  showstringspaces=false,
  columns=flexible,
  basicstyle={\small\ttfamily},
  numbers=none,
  numberstyle=\tiny\color{gray},
  keywordstyle=\color{blue},
  commentstyle=\color{dkgreen},
  stringstyle=\color{mauve},
  breaklines=true,
  breakatwhitespace=true,
  tabsize=3
}

\setlength\parindent{0pt} % Removes all indentation from paragraphs

\renewcommand{\labelenumi}{\alph{enumi}.} % Make numbering in the enumerate environment by letter rather than number (e.g. section 6)

% \title{Determination of the Atomic \\ Weight of Magnesium \\ CHEM 101} % Title
\title{Incremental Computing in Haskell\\CS240H Project Report}

\author[]{Jiyue \textsc{Wang}} 
\author[]{Kaixi \textsc{Ruan}}
\affil[]{}

\date{\today}

\begin{document}

\maketitle 

%----------------------------------------------------------------------------------------
% Introduction
%----------------------------------------------------------------------------------------
\section{Introduction}

%----------------------------------------------------------------------------------------
% Different concepts in the library, how to use it
%----------------------------------------------------------------------------------------
\section{Incremental in a nutshell }

\subsection{Incremental DAG}

% Use a demo to illustrate the following concepts.
\subsection{Demo}
\begin{description}
  \item [Variable] user could create variables which they could later change value 
  \item [Operation] like map/bind/arrayfold/...
  \item [Observer] use observer to observe some node/make it necessary
  \item [Stabilize] after building the graph/change the value, use stabilize to ...
  \item [Garbage Collection] 
\end{description}
 
% \begin{figure}[h]
% \begin{center}
% \includegraphics[width=0.65\textwidth]{placeholder} % Include the image placeholder.png
% \caption{Figure caption.}
% \end{center}
% \end{figure}


%----------------------------------------------------------------------------------------
%	Implementation 
%----------------------------------------------------------------------------------------
\section{Implementation}
\subsection{Node}
\begin{lstlisting}
  -- comment
  data Node a = Node {
    _kind      :: Kind a
  , _value     :: ValueInfo a
  , _edges     :: Edges
  }  
\end{lstlisting}

\subsection{Kind}

\subsection{State}

\subsection{Stabilization}
% TODO: add demo of async stabilization

\subsection{Observer}

%----------------------------------------------------------------------------------------
%	BIBLIOGRAPHY
%----------------------------------------------------------------------------------------

\bibliographystyle{apalike}

\bibliography{sample}

%----------------------------------------------------------------------------------------


\end{document}
